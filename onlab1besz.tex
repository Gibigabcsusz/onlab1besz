\documentclass[aspectratio=169]{beamer}

\usepackage[utf8]{inputenc} % mindenképp maradjon az utf-8 kódolás
\usepackage[magyar]{babel}
\usepackage[T1]{fontenc}
\usepackage{amsmath}
\usepackage{amsfonts}
\usepackage{amssymb}
\usepackage{graphics} % grafikus elemek, képek berakásához
\usepackage{blindtext}
%\usepackage{hyperref} % PDF hivatkozásokhoz kell
\usepackage[hang]{caption}
\usepackage{xcolor}
%\usepackage[affil-it]{authblk}

%\definecolor{rosewood}{rgb}{0.6, 0.0, 0.04}
%\definecolor{indigo(dye)}{rgb}{0.0, 0.25, 0.42}

\usetheme{Madrid}	% téma
\usefonttheme{serif}	% gyönyörű talpas betűtípus

\beamertemplatenavigationsymbolsempty % a pdf-be ágyazott navigációs gombok kikapcsolása

\title{2,4 GHz-es nyomtattott BIFA tervezése}			% cím
\subtitle{alcím} 	% alcím

\date{\today}

\author{Szilágyi Gábor}	% szerző

\institute{} % intézmény vagy más infó a szerzőről

\begin{document}
\maketitle	% címoldal
\begin{frame}
	\frametitle{Részfeladatok}
	\begin{columns}	
		\column{0.48\textwidth}
			%
		\column{0.48\textwidth}
			\begin{figure}
				\includegraphics[draft]{}
			\end{figure}
	\end{columns}
\end{frame}
% \begin{frame}
	% \frametitle{Részfeladatok}
	% %\framesubtitle{Az 1. dia alcíme}
	% \begin{columns}	
		% \column{0.48\textwidth}
			% \blindtext
		% \column{0.48\textwidth}
			% Lehetségesek ilyen átalakulások ütközések hatására:
			% \begin{align*}
				% Xe^+ + e^- \quad \longleftrightarrow \quad Xe
			% \end{align*}		
			% A töltött részecskék tere és a külső elektromágneses tér együtt hat a töltött részecskék mozgására.
			% \begin{align*}
				% \Sigma Q = 0
			% \end{align*}
	% \end{columns}
% \end{frame}
% \begin{frame}
	% \frametitle{Az egydimenziós modell}
	% \begin{columns}
		% \column{0.48\textwidth}
			% A következő egyszerűsítésekkel jutunk 3D-ből az 1D plazmához: \\
			% \begin{itemize}
				% \item A töltetlen $Xe$ részecskéket elhagyjuk
				% \item Az ütközésektől eltekintünk
				% \item Csak az elektronok mozgását vizsgáljuk
				% \item Pontszerű részecskék helyett felületi töltéssűrűséggel rendelkező, az $x$ tengelyre merőleges lapok
				% \item A pozitív töltésű $Xe^+$ ionokat helyhez kötött háttér-töltéssűrűségnek vesszük
			% \end{itemize}
		% \column{0.48\textwidth}
			% \begin{itemize}
				% \item A szimulációs tér egydimenziós és ciklikus, $x=0 \Longleftrightarrow x=N_g$
				% \item A külső elektromos teret 0-nak vesszük
				% \item A mégneses térnek nincs hatása 1D-ben
			% \end{itemize}
	% \end{columns}
% \end{frame}
% \begin{frame}
	% \frametitle{A Particle-Mesh módszer}
	% \begin{columns}
		% \column{0.48\textwidth}
			% Egy kis random szöveg
		% \column{0.48\textwidth}
			% \begin{figure}
				% \includegraphics[draft]{/home/g/Pictures/nap.png}
			% \end{figure}
	% \end{columns}
% \end{frame}
\end{document}
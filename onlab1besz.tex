\documentclass[aspectratio=43]{beamer}

\usepackage[utf8]{inputenc} % mindenképp maradjon az utf-8 kódolás
\usepackage[magyar]{babel}
\usepackage[T1]{fontenc}
\usepackage{amsmath}
\usepackage{amsfonts}
\usepackage{amssymb}
\usepackage{graphics} % grafikus elemek, képek berakásához
\usepackage{blindtext}
%\usepackage{hyperref} % PDF hivatkozásokhoz kell
\usepackage[hang]{caption}
\usepackage{xcolor}
\usepackage{siunitx}
%\usepackage[affil-it]{authblk}

%\definecolor{rosewood}{rgb}{0.6, 0.0, 0.04}
%\definecolor{indigo(dye)}{rgb}{0.0, 0.25, 0.42}

\usetheme{Boadilla}	% téma
\usefonttheme{serif}	% gyönyörű talpas betűtípus
\setbeamertemplate{footline}[frame number]

\setbeamercovered{transparent=30}
\beamertemplatenavigationsymbolsempty % a pdf-be ágyazott navigációs gombok kikapcsolása

\title{2,4 GHz-es nyomtattott BIFA tervezése}			% cím
\subtitle{} 	% alcím

\date{\today}

\author{Szilágyi Gábor}	% szerző

\institute{Tanszéki konzulens: \quad Dr. Lénárt Ferenc \\
	Ipari Konzulens (Silabs): \quad Bódi Tamás} % intézmény vagy más infó a szerzőről

\begin{document}
\maketitle	% címoldal
\begin{frame}
	\frametitle{Részfeladatok, koncepció}
	\begin{columns}
		\column{0.6\textwidth}
			\begin{enumerate}
				\item<0-3> Szimmetrikus tápvonal \\[1.5ex]
				\item<0-3> Szimmetrikus IFA (BIFA) \\[1.5ex]
				\item<0-2> Nyomtatott balun transzformátor \\[1.5ex]
				\item<0-1> Áramblokkoló mintázat
			\end{enumerate}
		\column{0.4\textwidth}
			\includegraphics[width=0.7\textwidth]{koncepcio.pdf}
	\end{columns}
\end{frame}
\begin{frame}
	\frametitle{Szimmetrikus tápvonal}
	\framesubtitle{CPS -- coplanar strip}
	\centering
	\includegraphics[width=\textwidth]{e1_2.png}
\end{frame}
\begin{frame}
	\frametitle{Antenna variációk}
		\centering
		\includegraphics[width=0.45\textwidth]{bifa_3D.png}
		\includegraphics[width=0.45\textwidth]{bifa_meandered_3D.png}
		\includegraphics[width=0.45\textwidth]{bifa_broadband_3D.png}
\end{frame}
\begin{frame}
	\frametitle{Antenna $Q$}
	\begin{columns}
		\column{0.48\textwidth}
			Közelítések $Q$-ra:\\
			\begin{align*}
				Q_1(\omega) & \approx \frac{\psi}{B_{V}(\omega)} \\
				\vspace{2cm}\\
				Q_2(\omega) & \approx \frac{\omega}{2 R(\omega)}|Z'(\omega)|
			\end{align*}
		\column{0.48\textwidth}
			\includegraphics[width=\textwidth]{bifa_broadband_QZ.png}
	\end{columns}
\end{frame}
\begin{frame}
	\frametitle{Antenna $S_{11}$}
		\centering
		\includegraphics[width=0.8\textwidth]{bifa_broadband_S11_dB.png}
\end{frame}
\begin{frame}
	\frametitle{Antenna iránykarakterisztika}
		\begin{columns}
			\column{0.3\textwidth}
				\centering
				\includegraphics[width=\textwidth]{lying_bifa_bb_3D.png}
			\column{0.68\textwidth}
				\centering
				\includegraphics[width=0.8\textwidth]{bifa_broadband_pattern_theta90.png}
		\end{columns}
\end{frame}
\begin{frame}
	\frametitle{Balun transzformátor}
		\centering
		Impedanciatranszformáció: \SI{50}{\ohm} --- \SI{100}{\ohm} \\
		Legjobb $S_{11}$ = -9,2 dB \\
		\vspace{\fill}
		\begin{columns}
			\column{0.4\textwidth}
				\centering
				\includegraphics[width=\textwidth]{cpw.pdf}
				\includegraphics[width=\textwidth]{cps.pdf}
			\column{0.6\textwidth}
				\centering
				\includegraphics[width=0.3\textwidth]{balun_1.png}
				\includegraphics[width=0.3\textwidth]{balun_2.png}
				\includegraphics[width=0.3\textwidth]{balun_3.png}
		\end{columns}
\end{frame}
\begin{frame}
		\begin{columns}
		\centering
			\column{0.3\textwidth}
				\centering
				\includegraphics[width=0.6\textwidth]{koncepcio.pdf}\\[3ex]
				\includegraphics[width=\textwidth]{cpw.pdf}
			\column{0.4\textwidth}
				\centering
				\includegraphics[width=\textwidth]{e1_2.png}\\[3ex]
				\includegraphics[width=0.6\textwidth]{bifa_broadband_3D.png}
				\includegraphics[width=0.28\textwidth]{balun_3.png}\\[3ex]
				\includegraphics[width=0.6\textwidth]{lying_bifa_bb_3D.png}
			\column{0.3\textwidth}
				\includegraphics[width=\textwidth]{bifa_broadband_QZ.png}
				\includegraphics[width=\textwidth]{bifa_broadband_S11_dB.png}
				\includegraphics[width=\textwidth]{bifa_broadband_pattern_theta90.png}
		\end{columns}
\end{frame}
\end{document}